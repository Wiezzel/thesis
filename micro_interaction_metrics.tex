\begin{longtable}{D{0.2} c X}
\caption{Micro-interaction metrics} 
\label{tab:micro_interaction_metrics}\\
\toprule
\textbf{Level} & \textbf{Category} & \multicolumn{1}{C}{\textbf{Variable}} \\
\midrule
\multirow{6}{*}{File}
& \multirow{2}{*}{Effort}
& Number of selection events of the file\\
& & Number of edit events of the file \\
\cline{2-3}
& \multirow{2}{*}{Interest}
& Average \emph{degree of interest}\footnote{\emph{Degree of interest} (DOI) is a parameter increasing on programmer's interactions and decreasing with time, thus measuring the interest in a particular file or code element. It was introduced by Kersten \& Murhpy in \cite{mylar} together with an implementation named \textsc{Mylar}.} for the file\\
& & Variance of \emph{degree of interest} for the file\\
\cline{2-3}
& \multirow{2}{*}{Intervals}
& Average time interval between edit events  of the file\\
& & Average time interval between selection events of the file \\
\midrule
\multirow{13}{*}{Task}
& \multirow{3}{*}{Effort}
& Total time spent on the task \\
& & Number of unique selected files during the task \\
& & Number of unique edited files during the task \\
\cline{2-3}
& \multirow{2}{*}{Distraction}
& Number of selection events with low \emph{degree of interest} \\
& & Number of edit events with low \emph{degree of interest} \\
\cline{2-3}
& \multirow{4}{*}{\cpbox{5cm}{Work \\ portion}}
& Ratio of time spent (out of total time) before the first edit \\
& & Ratio of time spent (out of total time) after the last edit \\
& & Number of unique selections before the first edit \\
& & Number of unique selections after the last edit \\
\cline{2-3}
& \multirow{2}{*}{Repetition}
& Number of files selected more than once in the task \\
& & Number of files edited more than once in the task \\
\cline{2-3}
& \multirow{2}{*}{Task load}
& Number of ongoing tasks at the same time \\
& & Average depth of the file hierarchy \\
\bottomrule
\end{longtable}